%package list
\documentclass{article}
\usepackage[top=3cm, bottom=3cm, outer=3cm, inner=3cm]{geometry}
\usepackage{multicol}
\usepackage{graphicx}
\usepackage{url}
%\usepackage{cite}
\usepackage{hyperref}
\usepackage{array}
%\usepackage{multicol}
\newcolumntype{x}[1]{>{\centering\arraybackslash\hspace{0pt}}p{#1}}
\usepackage{natbib}
\usepackage{pdfpages}
\usepackage{multirow}
\usepackage[normalem]{ulem}
\useunder{\uline}{\ul}{}
\usepackage{svg}
\usepackage{xcolor}
\usepackage{listings}
\lstdefinestyle{ascii-tree}{
    literate={├}{|}1 {─}{--}1 {└}{+}1 
  }
\lstset{basicstyle=\ttfamily,
  showstringspaces=false,
  commentstyle=\color{red},
  keywordstyle=\color{blue}
}
%\usepackage{booktabs}
\usepackage{caption}
\usepackage{subcaption}
\usepackage{float}
\usepackage{array}

\newcolumntype{M}[1]{>{\centering\arraybackslash}m{#1}}
\newcolumntype{N}{@{}m{0pt}@{}}


%%%%%%%%%%%%%%%%%%%%%%%%%%%%%%%%%%%%%%%%%%%%%%%%%%%%%%%%%%%%%%%%%%%%%%%%%%%%
%%%%%%%%%%%%%%%%%%%%%%%%%%%%%%%%%%%%%%%%%%%%%%%%%%%%%%%%%%%%%%%%%%%%%%%%%%%%
\newcommand{\itemEmail}{rcompanocca@unsa.edu.pe}
\newcommand{\itemStudent}{Roni Companocca Checco}
\newcommand{\itemCourse}{Programación}
\newcommand{\itemCourseCode}{20210558}
\newcommand{\itemSemester}{II}
\newcommand{\itemUniversity}{Universidad Nacional de San Agustín de Arequipa}
\newcommand{\itemFaculty}{Facultad de Ingeniería de Producción y Servicios}
\newcommand{\itemDepartment}{Departamento Académico de Ingeniería de Sistemas e Informática}
\newcommand{\itemSchool}{Escuela Profesional de Ingeniería de Sistemas}
\newcommand{\itemAcademic}{2023 - B}
\newcommand{\itemInput}{Del 29 Septiembre 2023}
\newcommand{\itemOutput}{Al 30 Septiembre 2023}
\newcommand{\itemPracticeNumber}{06}
\newcommand{\itemTheme}{HashMap}
%%%%%%%%%%%%%%%%%%%%%%%%%%%%%%%%%%%%%%%%%%%%%%%%%%%%%%%%%%%%%%%%%%%%%%%%%%%%
%%%%%%%%%%%%%%%%%%%%%%%%%%%%%%%%%%%%%%%%%%%%%%%%%%%%%%%%%%%%%%%%%%%%%%%%%%%%

\usepackage[english,spanish]{babel}
\usepackage[utf8]{inputenc}
\AtBeginDocument{\selectlanguage{spanish}}
\renewcommand{\figurename}{Figura}
\renewcommand{\refname}{Referencias}
\renewcommand{\tablename}{Tabla} %esto no funciona cuando se usa babel
\AtBeginDocument{%
	\renewcommand\tablename{Tabla}
}

\usepackage{fancyhdr}
\pagestyle{fancy}
\fancyhf{}
\setlength{\headheight}{30pt}
\renewcommand{\headrulewidth}{1pt}
\renewcommand{\footrulewidth}{1pt}
\fancyhead[L]{\raisebox{-0.2\height}{\includegraphics[width=3cm]{logo_episunsa.png}}}
\fancyhead[C]{\fontsize{7}{7}\selectfont	\itemUniversity \\ \itemFaculty \\ \itemDepartment \\ \itemSchool \\ \textbf{\itemCourse}}
\fancyhead[R]{\raisebox{-0.2\height}{\includegraphics[width=1.2cm]{abet.png}}}
\fancyfoot[L]{Estudiante Roni Companocca Checco}
\fancyfoot[C]{\itemCourse}
\fancyfoot[R]{Página \thepage}

% para el codigo fuente
\usepackage{listings}
\usepackage{color, colortbl}
\definecolor{dkgreen}{rgb}{0,0.6,0}
\definecolor{gray}{rgb}{0.5,0.5,0.5}
\definecolor{mauve}{rgb}{0.58,0,0.82}
\definecolor{codebackground}{rgb}{0.95, 0.95, 0.92}
\definecolor{tablebackground}{rgb}{0.8, 0, 0}

\lstset{frame=tb,
	language=bash,
	aboveskip=3mm,
	belowskip=3mm,
	showstringspaces=false,
	columns=flexible,
	basicstyle={\small\ttfamily},
	numbers=none,
	numberstyle=\tiny\color{gray},
	keywordstyle=\color{blue},
	commentstyle=\color{dkgreen},
	stringstyle=\color{mauve},
	breaklines=true,
	breakatwhitespace=true,
	tabsize=3,
	backgroundcolor= \color{codebackground},
}

\begin{document}
	
	\vspace*{10px}
	
	\begin{center}	
		\fontsize{17}{17} \textbf{ Informe de Laboratorio \itemPracticeNumber}
	\end{center}
	\centerline{\textbf{\Large Tema: \itemTheme}}
	%\vspace*{0.5cm}	

	\begin{flushright}
		\begin{tabular}{|M{2.5cm}|N|}
			\hline 
			\rowcolor{tablebackground}
			\color{white} \textbf{Nota}  \\
			\hline 
			     \\[30pt]
			\hline 			
		\end{tabular}
	\end{flushright}	

	\begin{table}[H]
		\begin{tabular}{|x{4.7cm}|x{4.8cm}|x{4.8cm}|}
			\hline 
			\rowcolor{tablebackground}
			\color{white} \textbf{Estudiante} & \color{white}\textbf{Escuela}  & \color{white}\textbf{Asignatura}   \\
			\hline 
			{\itemStudent \par \itemEmail} & \itemSchool & {\itemCourse \par Semestre: \itemSemester \par Código: \itemCourseCode}     \\
			\hline 			
		\end{tabular}
	\end{table}		
	
	\begin{table}[H]
		\begin{tabular}{|x{4.7cm}|x{4.8cm}|x{4.8cm}|}
			\hline 
			\rowcolor{tablebackground}
			\color{white}\textbf{Laboratorio} & \color{white}\textbf{Tema}  & \color{white}\textbf{Duración}   \\
			\hline 
			\itemPracticeNumber & \itemTheme & 02 horas   \\
			\hline 
		\end{tabular}
	\end{table}
	
	\begin{table}[H]
		\begin{tabular}{|x{4.7cm}|x{4.8cm}|x{4.8cm}|}
			\hline 
			\rowcolor{tablebackground}
			\color{white}\textbf{Semestre académico} & \color{white}\textbf{Fecha de inicio}  & \color{white}\textbf{Fecha de entrega}   \\
			\hline 
			\itemAcademic & \itemInput &  \itemOutput  \\
			\hline 
		\end{tabular}
	\end{table}

    \section{TAREA}
	\begin{itemize}	
    \subsection{Objetivos:}
		\item Crear e inicializar HashMaps
		\item Realizar búsquedas secuencial y binaria en un HashMap
        \item Implementar métodos de ordenamiento en HashMap
        \item Solucionar problemas
    \subsection{Competencias a alcanzar:}
		\item Diseña, responsablemente, sistemas, componentes o procesos para satisfacer necesidades dentro de restricciones realistas: económicas, medio ambientales, sociales, políticas, éticas, de salud, de seguridad, manufacturación y sostenibilidad.
        \item Aplica de forma flexible, técnicas, métodos, principios, normas, estándares y herramientas de ingeniería necesarias para la construcción de software e implementación de sistemas de información.

    \section{EQUIPOS, MATERIALES Y TEMAS UTILIZADOS}
	\begin{itemize}
		\item Sistema Operativo Windows
		\item OpenJDK 64-Bits 17.0.7.
		\item Git 2.39.2.	
  	\item Cuenta en GitHub con el correo institucional.
	\end{itemize}

    \section{URL DE REPOSITORIO GITHUB}
	\begin{itemize}
		\item URL para el Repositorio GitHub.
		\item \url{https://github.com/RONI-COMPANOCCA-CHECCO}
		\item URL para el laboratorio 06 en el Repositorio GitHub.	
        \item \url{https://github.com/RONI-COMPANOCCA-CHECCO/FP2-LAB6}
	\end{itemize}
    
    \section{ACTIVIDADES}
	\begin{itemize}
        \subsection{EJERCICIO RESUELTO}
		\item - Creamos la Clase ColeccionesMap con los siguientes métodos que realizan las operaciones que se pueden realizar con el HashMap:

        \begin{lstlisting}[language=java]
// RONI COMPANOCCA CHECCO
// CUI: 20210558
// LABORATORIO 06
// FUNDAMENTOS DE PROGRAMACION 
// EJERCICIO PROPUESTO DESARROLLADO
import java.util.Map;
import java.util.HashMap;
import java.util.Iterator;
import java.util.Scanner;

public class ColeccionesMap{
    static Scanner consola = new Scanner(System.in);
    // interfaz implement
    Map diccionario1 = new HashMap();
    public void llenarHashMap(){
        // Insertar valores "key"-"value" al HashMap
        diccionario1.put("29458865","Maria");
        diccionario1.put("28784599","Jose");
        diccionario1.put("26558899","Luis");
        diccionario1.put("45785214","Juan");
        diccionario1.put("48889566","Mariela");
        diccionario1.put("48889544","Carlos");
        diccionario1.put("12389566","Rosa");
        diccionario1.put("48889540","Carlos");
        diccionario1.put("26558890","Luis");

        // definir iterator ara extraer o imprimir valores
        // recorremos y mostramos el diccionario
        for(Iterator iterador1 = diccionario1.keySet().iterator(); iterador1.hasNext();){
            String key = (String)iterador1.next();
            String value = (String)diccionario1.get(key);
            System.out.println("DNI: "+key+" Alumno: "+value);
        }
        // mstramos el tamaño del diccionario
        System.out.println("Tamaño: "+diccionario1.size());
    }

    public void buscarEnHashMap(){
        // buscamos un valor en el diccionario
        System.out.println("Ingrese un Nombre a Buscar: ");
        String nombre = consola.next();
        if(diccionario1.containsValue(nombre)){
            System.out.println((nombre+" se encuentra en el Diccionario"));
        }else{
            System.out.println(nombre+ " No se encuentra!!");
        }
    }

    public void eliminarEnHashMap(){
        // eliminamos un valor en el diccionario
        System.out.println("Ingrese un Codigo a Eliminar: ");
        String llave = consola.next();
        diccionario1.remove( llave);

        for(Object key : diccionario1.keySet()){
            String llave2 = (String)key;
            String value2 = (String)diccionario1.get(llave2);
            System.out.println("DNI: "+llave2+" Alumno: "+value2);
        }
    }

    public void reemplazarEnHashMap(){
        // reemplazamos un valor en el diccionario
        System.out.println("Ingrese un codigo a reemplazar: ");
        String llave2 = consola.next();
        System.out.println("Ingrese el Nuevo Valor: ");
        String valorNuevo = consola.next();
        diccionario1.replace(llave2, valorNuevo);
        
        for(Object key3 : diccionario1.keySet()){
            String llave3 = (String)key3;
            String value3 = (String)diccionario1.get(llave3);
            System.out.println(("DNI: "+llave3+" Alumno: "+value3));
        } 
    }
}
        \end{lstlisting}
		\item Creamos la Clase Main, donde llamamos a nuestra clase y las operaciones implementadas:
        \begin{lstlisting}[language=java]
public class Map01 {
    public static void main(String[] args){
        ColeccionesMap coleccion01 = new ColeccionesMap();
        System.out.println("----------------------");
        System.out.println("COLECCION HASHMAP");
        System.out.println("----------------------");
        coleccion01.llenarHashMap();
        coleccion01.buscarEnHashMap();
        coleccion01.reemplazarEnHashMap();
        coleccion01.eliminarEnHashMap();
    }
}
        \end{lstlisting}
        \item Ejecucion: Ejecutamos el ejercicio desarrollado por la cual nos como respuesta lo siguiente.
        \begin{lstlisting}[language=java]
----------------------
COLECCION HASHMAP
----------------------
DNI: 26558899 Alumno: Luis
DNI: 48889540 Alumno: Carlos
DNI: 28784599 Alumno: Jose
DNI: 29458865 Alumno: Maria
DNI: 48889566 Alumno: Mariela
DNI: 48889544 Alumno: Carlos
DNI: 12389566 Alumno: Rosa
DNI: 26558890 Alumno: Luis
DNI: 45785214 Alumno: Juan
Tamaño: 9
Ingrese un Nombre a Buscar:
Carlos
Carlos se encuentra en el Diccionario
Ingrese un codigo a reemplazar:
48889544
Ingrese el Nuevo Valor:
Roni
DNI: 26558899 Alumno: Luis
DNI: 48889540 Alumno: Carlos
DNI: 28784599 Alumno: Jose
DNI: 29458865 Alumno: Maria
DNI: 48889566 Alumno: Mariela
DNI: 48889544 Alumno: Roni
DNI: 12389566 Alumno: Rosa
DNI: 26558890 Alumno: Luis
DNI: 45785214 Alumno: Juan
Ingrese un Codigo a Eliminar:
48889544
DNI: 26558899 Alumno: Luis
DNI: 48889540 Alumno: Carlos
DNI: 28784599 Alumno: Jose
DNI: 29458865 Alumno: Maria
DNI: 48889566 Alumno: Mariela
DNI: 12389566 Alumno: Rosa
DNI: 26558890 Alumno: Luis
DNI: 45785214 Alumno: Juan
        \end{lstlisting}

        \subsection{EJERCICIO PROPUESTO}
        \item Cree un Proyecto llamado Laboratorio6
        \item Usted deberá crear las dos clases Soldado.java y VideoJuego5.java. Puede reutilizar lo desarrollado en Laboratorios anteriores.
        \item Del Soldado nos importa el nombre, puntos de vida, fila y columna (posición en el tablero).
        \item El juego se desarrollará en el mismo tablero de los laboratorios anteriores. Para crear el tablero utilice la estructura de datos más adecuada.
        \item Tendrá 2 Ejércitos (usar HashMaps). Inicializar el tablero con n soldados aleatorios entre 1 y 10 para cada Ejército. Cada soldado tendrá un nombre autogenerado: Soldado0X1, Soldado1X1, etc., un valor de puntos de vida autogenerado aleatoriamente [1..5], la fila y columna también autogenerados aleatoriamente (no puede haber 2 soldados en el mismo cuadrado). Se debe mostrar el tablero con todos los soldados creados (distinguir los de un ejército de los del otro ejército). Además de los datos del Soldado con mayor vida de cada ejército, el promedio de puntos de vida de todos los soldados creados por ejército, los datos de todos los soldados por ejército en el orden que fueron creados y un ranking de poder de todos los soldados creados por ejército (del que tiene más nivel de vida al que tiene menos) usando 2 diferentes algoritmos de ordenamiento (indicar conclusiones respecto a este ordenamiento de HashMaps). Finalmente, que muestre qué ejército ganará la batalla (indicar la métrica usada para decidir al ganador de la batalla). Hacerlo como programa iterativo.
        
        \item la clase VideoJuego5.java        
        \begin{lstlisting}[language=java]
// RONI COMPANOCCA CHECCO
// CUI: 20210558
// LABORATORIO 06 - HASHMAP
// FUNDAMENTOS DE PROGRAMACION 
import java.util.*;

public class VideoJuego5 {

    public static void main(String[] args) {

        // DECLARACIÓN DE VARIABLES Y ESTRUCTURAS DE DATOS NECESARIAS
        HashMap<String, Soldado> ejercito1 = new HashMap<>();
        HashMap<String, Soldado> ejercito2 = new HashMap<>();
        HashMap<String, Soldado> tablero = new HashMap<>();
        int batallon1, batallon2;
        int vidatotal1 = 0, vidatotal2 = 0;
        double promedioVida1 = 0, promedioVida2 = 0;

        // BUCLE PARA DESIGNAR LA CANTIDAD DE FILAS Y COLUMNAS DEL TABLERO
        for (int i = 1; i <= 10; i++) {
            for (int j = 1; j <= 10; j++) {
                String posicion = i + "x" + j;
                tablero.put(posicion, new Soldado());
            }
        }

        // CREACIÓN DEL NÚMERO DE POSICIONES DE CADA EJÉRCITO
        batallon1 = aleatorio(1, 10);
        batallon2 = aleatorio(1, 10);

        // INICIALIZAR MAPAS DE EJÉRCITOS
        inicializarEjercito(ejercito1, batallon1);
        inicializarEjercito(ejercito2, batallon2);

        // GENERAR EJÉRCITOS VÁLIDOS
        generarEjercitos(ejercito1, ejercito2, tablero);

        // IMPRIMIR EL TABLERO
        imprimirTablero(tablero);

        // IMPRIMIR LOS SOLDADOS DE MAYOR VIDA DE CADA EJÉRCITO
        System.out.println("Soldado de mayor vida del ejército 1");
        SoldadoConMayorVida(ejercito1);
        System.out.println("Soldado de mayor vida del ejército 2");
        SoldadoConMayorVida(ejercito2);

        // IMPRIMIR LA VIDA TOTAL Y EL PROMEDIO DEL EJÉRCITO 1
        System.out.println("\nEJÉRCITO 1: ");
        for (Soldado soldado : ejercito1.values()) {
            vidatotal1 += soldado.getPuntos();
        }
        promedioVida1 = vidatotal1 / (double) ejercito1.size();
        System.out.println("Vida total: " + vidatotal1);
        System.out.println("Promedio de vida: " + promedioVida1);

        // IMPRIMIR LA VIDA TOTAL Y EL PROMEDIO DEL EJÉRCITO 2
        System.out.println("\nEJÉRCITO 2: ");
        for (Soldado soldado : ejercito2.values()) {
            vidatotal2 += soldado.getPuntos();
        }
        promedioVida2 = vidatotal2 / (double) ejercito2.size();
        System.out.println("Vida total: " + vidatotal2);
        System.out.println("Promedio de vida: " + promedioVida2);

        // IMPRIMIR LOS SOLDADOS CREADOS EN EL ORDEN POR DEFECTO
        System.out.println("\nLista ejército 1:");
        for (Soldado soldado : ejercito1.values()) {
            imprimir(soldado);
        }
        System.out.println("\nLista ejército 2:");
        for (Soldado soldado : ejercito2.values()) {
            imprimir(soldado);
        }

        // IMPRIMIR LOS DATOS DE LOS SOLDADOS ORDENADOS DE MAYOR A MENOR DEPENDIENDO DE SU NIVEL DE VIDA
        ArrayList<Soldado> listaSoldados1 = new ArrayList<>(ejercito1.values());
        ArrayList<Soldado> listaSoldados2 = new ArrayList<>(ejercito2.values());
        ordenarPorVidaMetodoA(ejercito1);
        ordenarPorVidaMetodoB(ejercito2);
        System.out.println("\nEjército 1 Ordenados por nivel de vida");
        for (Soldado soldado : listaSoldados1) {
            imprimir(soldado);
        }
        System.out.println("\nEjército 2 Ordenados por nivel de vida");
        for (Soldado soldado : listaSoldados2) {
            imprimir(soldado);
        }

        // MOSTRAR EJÉRCITO GANADOR LA MÉTRICA USADA PARA DESIGNAR AL GANADOR ES EL PROMEDIO DEL NIVEL DE VIDA DE CADA EJÉRCITO
        if (promedioVida1 > promedioVida2) {
            System.out.println("\nGANADOR ***EJÉRCITO 1***");
        } else if (promedioVida1 < promedioVida2) {
            System.out.println("\nGANADOR ***EJÉRCITO 2***");
        } else {
            System.out.print("\n***ES UN EMPATE***");
        }
    }

    // METODO PARA CREAR NUMEROS ALEATORIOS EN UN RANGO
	public static int aleatorio(int min, int max) {
		return(int)(Math.random()*(max-min+1)+min);
	}
    
    // METODO PARA INICIAR UN EJÉRCITO
    public static void inicializarEjercito(HashMap<String, Soldado> ejercito, int num) {
        for (int i = 0; i < num; i++) {
            ejercito.put("Soldado" + i, new Soldado());
        }
    }

	// METODO PARA GENERAR DATOS DEL OBJETO SOLDADO
	public static Soldado generarDatos() {
		Soldado soldadito = new Soldado();
		soldadito.setPuntos(aleatorio(1,5));
		soldadito.setColumna(aleatorio(1,10));
		soldadito.setFila(aleatorio(1,10));
		return soldadito;
	}

	// METODOS PARA GENERAR LOS EJERCITOS DE MANERA ALEATORIA
    public static void generarEjercitos(HashMap<String, Soldado> ejercito1, HashMap<String, Soldado> ejercito2, HashMap<String, Soldado> tablero) {
        ArrayList<String> posicionesOcupadas = new ArrayList<>();
        
        // Generar los soldados y asegurarse de que sus posiciones sean únicas
        for (int i = 0; i < ejercito1.size() + ejercito2.size(); i++) {
            Soldado soldado = generarDatos();
            String posicion = soldado.getFila() + "x" + soldado.getColumna();
            
            // Verificar que la posición no esté ocupada
            while (posicionesOcupadas.contains(posicion)) {
                soldado = generarDatos();
                posicion = soldado.getFila() + "x" + soldado.getColumna();
            }
            
            posicionesOcupadas.add(posicion);
            
            // Asignar soldado a ejército 1 o ejército 2
            if (i < ejercito1.size()) {
                ejercito1.put("Soldado" + i + "x1", soldado);
            } else {
                ejercito2.put("Soldado" + (i - ejercito1.size()) + "x2", soldado);
            }
            
            // Asignar soldado al tablero
            tablero.put(posicion, soldado);
        }
        
        // Actualizar las columnas de los soldados en ejército 1 y ejército 2
        for (String clave : ejercito1.keySet()) {
            Soldado soldado = ejercito1.get(clave);
            soldado.setColumn(soldado.getPuntos() + "[E1]");
        }
        for (String clave : ejercito2.keySet()) {
            Soldado soldado = ejercito2.get(clave);
            soldado.setColumn(soldado.getPuntos() + "[E2]");
        }
    }

    // METODO PARA AÑADIR LOS EJÉRCITOS AL TABLERO
    public static void añadirTablero(HashMap<String, Soldado> ejercito, HashMap<String, Soldado> tablero) {
        for (String posicion : ejercito.keySet()) {
            tablero.put(posicion, ejercito.get(posicion));
        }
    }

    // METODO PARA IMPRIMIR EL TABLERO EN LA CUAL SE DESARROLLA EL JUEGO
    public static void imprimirTablero(HashMap<String, Soldado> tablero) {
        System.out.println("\tA\tB\tC\tD\tF\tG\tH\tI\tJ");
        for (int i = 1; i <= 10; i++) {
            System.out.print(i);
            for (int j = 1; j <= 10; j++) {
                String posicion = i + "x" + j;
                Soldado soldado = tablero.get(posicion);
                System.out.print("\t" + soldado.getColumn());
            }
            System.out.println("\n");
        }
    }

    //METODO PARA IMPRIMIR LOS SOLDADOS DE MAYOR VIDA
	public static void SoldadoConMayorVida(HashMap<String, Soldado> soldados) {
        Soldado mayor = null;
    
        for (Soldado soldado : soldados.values()) {
            if (mayor == null || soldado.getPuntos() > mayor.getPuntos()) {
                mayor = soldado;
            }
        }
    
        if (mayor != null) {
            imprimir(mayor);
        } else {
            System.out.println("No se encontraron soldados.");
        }
    }
	
	// METODO PARA IMPRIMIR EL NOMBRE, LA POSICION Y NIVEL DE VIDA DEL SOLDADO
	public static void imprimir(Soldado soldadito) {
		System.out.println("Nombre: "+soldadito.getNombre()+"\nPosicion: "+soldadito.getColumna()+"X"+soldadito.getFila()+"\tVida: "+soldadito.getPuntos());
	}
	
	// METODO QUE NOS AYUDA A ORDENAR LOS SOLDADOS DE ACUERDO A SU NIVEL DE VIDA, USUANDO UN ALGORITMO DE ORDENAMIENTO DE BURBUJA
	public static void ordenarPorVidaMetodoA(HashMap<String, Soldado> soldados) {
        // Obtener los valores (los soldados) del HashMap y almacenarlos en una lista
        List<Soldado> listaSoldados = new ArrayList<>(soldados.values());
    
        Soldado aux = new Soldado();
        for (int i = 0; i < listaSoldados.size() - 1; i++) {
            for (int j = 0; j < listaSoldados.size() - i - 1; j++) {
                if (listaSoldados.get(j).getPuntos() < listaSoldados.get(j + 1).getPuntos()) {
                    aux = listaSoldados.get(j);
                    listaSoldados.set(j, listaSoldados.get(j + 1));
                    listaSoldados.set(j + 1, aux);
                }
            }
        }
    
        // Actualizar el HashMap con los soldados ordenados
        int index = 0;
        for (String clave : soldados.keySet()) {
            soldados.put(clave, listaSoldados.get(index));
            index++;
        }
    }

    // METODO QUE NOS AYUDA A ORDENAR LOS SOLDADOS DE ACUERDO A SU NIVEL DE VIDA, EN ESTA OCACION DIFERENTE A LA ANTERIOR QUE ERA ALGORITMO DE BURBUJA
    public static void ordenarPorVidaMetodoB(HashMap<String, Soldado> soldados) {
        // Obtener los valores (los soldados) del HashMap y almacenarlos en una lista
        List<Soldado> listaSoldados = new ArrayList<>(soldados.values());
    
        // Ordenar la lista en orden descendente por puntos de vida
        Collections.sort(listaSoldados, new Comparator<Soldado>() {
            public int compare(Soldado s1, Soldado s2) {
                // Orden descendente por puntos de vida
                return Integer.compare(s2.getPuntos(), s1.getPuntos());
            }
        });
    
        // Actualizar el HashMap con los soldados ordenados
        int index = 0;
        for (String clave : soldados.keySet()) {
            soldados.put(clave, listaSoldados.get(index));
            index++;
        }
    }
}
        \end{lstlisting}

        \item la clase Soldado.java
        \begin{lstlisting}[language=java]
// RONI COMPANOCCA CHECCO
// CUI: 20210558
// LABORATORIO 06
// FUNDAMENTOS DE PROGRAMACION - LABORATORIO
public class Soldado {
	private String nombre;
	private int fila;
	private int columna;
	private int puntos;
	private String column;
	
	public Soldado() {
		nombre = "";
		fila = 0;
		columna = 0;
		puntos = 0;
		column = "";
	}

	// METODOS MUTADORES
	public void setNombre(String nombre) {
		this.nombre = nombre;
	}

	public void setFila(int fila) {
		this.fila = fila;
	}

	public void setColumna(int columna) {
		this.columna = columna;
	}

	public void setPuntos(int puntos) {
		this.puntos = puntos;
	}
	
	public void setColumn(String column) {
		this.column = column;
	}

	// METODOS ACCESORES
	public String getNombre() {
		return nombre;
	}

	public int getFila() {
		return fila;
	}

	public int getColumna() {
		return columna;
	}

	public int getPuntos() {
		return puntos;
	}
	
	public String getColumn() {
		return column;
	}
}
         \end{lstlisting}

         \item Ejecucion
         \begin{lstlisting}[language=java]
        A       B       C       D       F       G       H       I       J
1                       4[E2]   4[E2]

2       1[E1]                                           3[E2]

3               5[E2]   1[E1]                                           5[E1]

4       3[E2]

5                       2[E1]                   5[E1]

6       1[E2]   5[E2]

7

8                                                                       1[E1]

9                       4[E2]

10              4[E2]                   5[E1]   1[E2]

Soldado de mayor vida del ejercito 1
Nombre: Soldado0x1
Posicion: 10X5  Vida: 5
soldado de mayor vida del ejercito 2
Nombre: Soldado1x2
Posicion: 6X2   Vida: 5

EJERCITO 1:
Vida total: 20
Promedio de vida: 2.857142857142857

EJERCITO 2:
Vida total: 34
Promedio de vida: 3.4

Lista ejercito 1:
Nombre: Soldado0x1
Posicion: 10X5  Vida: 5
Nombre: Soldado1x1
Posicion: 8X9   Vida: 1
Nombre: Soldado2x1
Posicion: 3X3   Vida: 1
Nombre: Soldado3x1
Posicion: 5X6   Vida: 5
Nombre: Soldado4x1
Posicion: 5X3   Vida: 2
Nombre: Soldado5x1
Posicion: 2X1   Vida: 1
Nombre: Soldado6x1
Posicion: 3X9   Vida: 5

Lista ejercito 2:
Nombre: Soldado0x2
Posicion: 1X4   Vida: 4
Nombre: Soldado1x2
Posicion: 6X2   Vida: 5
Nombre: Soldado2x2
Posicion: 10X6  Vida: 1
Nombre: Soldado3x2
Posicion: 9X3   Vida: 4
Nombre: Soldado4x2
Posicion: 2X7   Vida: 3
Nombre: Soldado5x2
Posicion: 4X1   Vida: 3
Nombre: Soldado6x2
Posicion: 10X2  Vida: 4
Nombre: Soldado7x2
Posicion: 6X1   Vida: 1
Nombre: Soldado8x2
Posicion: 3X2   Vida: 5
Nombre: Soldado9x2
Posicion: 1X3   Vida: 4

Ejercito 1 Ordenados por nivel de vida
Nombre: Soldado0x1
Posicion: 10X5  Vida: 5
Nombre: Soldado3x1
Posicion: 5X6   Vida: 5
Nombre: Soldado6x1
Posicion: 3X9   Vida: 5
Nombre: Soldado4x1
Posicion: 5X3   Vida: 2
Nombre: Soldado1x1
Posicion: 8X9   Vida: 1
Nombre: Soldado2x1
Posicion: 3X3   Vida: 1
Nombre: Soldado5x1
Posicion: 2X1   Vida: 1

Ejercito 2 Ordenados por nivel de vida
Nombre: Soldado1x2
Posicion: 6X2   Vida: 5
Nombre: Soldado8x2
Posicion: 3X2   Vida: 5
Nombre: Soldado0x2
Posicion: 1X4   Vida: 4
Nombre: Soldado3x2
Posicion: 9X3   Vida: 4
Nombre: Soldado6x2
Posicion: 10X2  Vida: 4
Nombre: Soldado9x2
Posicion: 1X3   Vida: 4
Nombre: Soldado4x2
Posicion: 2X7   Vida: 3
Nombre: Soldado5x2
Posicion: 4X1   Vida: 3
Nombre: Soldado2x2
Posicion: 10X6  Vida: 1
Nombre: Soldado7x2
Posicion: 6X1   Vida: 1

GANADOR ***EJERCITO 2***
         \end{lstlisting}
 
    \section{CUESTIONARIO}

	\begin{itemize}
	\subsection {¿Cómo se declara e inicializa un HashMap?}
        \item Para declarar e inicializar un HashMap en Java
        \begin{lstlisting}[language=java]
        import java.util.HashMap;
        HashMap<KeyType, ValueType> nombreDelMapa = new HashMap<>();
        \end{lstlisting}
		
    \subsection {¿Qué ventajas tienen los HashMap con respecto a los arreglos y ArrayList?}
        \item Los HashMap en Java y otras estructuras de datos basadas en tablas de hash ofrecen varias ventajas en comparación con los arreglos (arrays) y ArrayLists como: Búsqueda eficiente, flexibilidad en las claves, capacidad dinámica, eliminación eficiente, no hay restricciones en la ubicación, claves únicas, etc.

	\subsection {¿Mencione 4 métodos importantes del HashMap? ¿Qué finalidad tienen?}
        \item put(Key key, Value value): Agregar un par clave-valor al HashMap. Si la clave ya existe en el mapa, el valor asociado se actualiza con el nuevo valor proporcionado. Si la clave no existe, se crea un nuevo par clave-valor en el mapa.
        \item get(Object key): Recuperar el valor asociado con una clave específica en el HashMap.
        \item remove(Object key): Eliminar el par clave-valor asociado con la clave proporcionada del HashMap.
        \item containsKey(Object key): Verificar si el HashMap contiene una clave específica. Devuelve true si la clave existe en el mapa y false en caso contrario.
	\end{itemize}
	
	\section{REFERENCIAS}
	\begin{itemize}
		\item M. Aedo, “Fundamentos de Programación 2 - Tópicos de Programación Orientada a Objetos”, Primera Edición, 2021, Editorial UNSA.
		\item \url{https://github.com/rescobedoq/programacion.git}
		\item J. Dean, "Introduction to programming with Java: A Problem Solving Approach”, Third Edition, 2021, McGraw-Hill.
        \item C. T. Wu, "An Introduction to Object-Oriented Programming with Java", Fifth Edition, 2010, McGraw-Hill.
        \item P. Deitel, "Java How to Program", Eleventh Edition, 2017, Prentice Hall.
	\end{itemize}
	
%\clearpage
%\bibliographystyle{apalike}
%\bibliographystyle{IEEEtranN}
%\bibliography{bibliography}
			
\end{document}