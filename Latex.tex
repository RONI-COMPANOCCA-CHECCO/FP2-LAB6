%package list
\documentclass{article}
\usepackage[top=3cm, bottom=3cm, outer=3cm, inner=3cm]{geometry}
\usepackage{multicol}
\usepackage{graphicx}
\usepackage{url}
%\usepackage{cite}
\usepackage{hyperref}
\usepackage{array}
%\usepackage{multicol}
\newcolumntype{x}[1]{>{\centering\arraybackslash\hspace{0pt}}p{#1}}
\usepackage{natbib}
\usepackage{pdfpages}
\usepackage{multirow}
\usepackage[normalem]{ulem}
\useunder{\uline}{\ul}{}
\usepackage{svg}
\usepackage{xcolor}
\usepackage{listings}
\lstdefinestyle{ascii-tree}{
    literate={├}{|}1 {─}{--}1 {└}{+}1 
  }
\lstset{basicstyle=\ttfamily,
  showstringspaces=false,
  commentstyle=\color{red},
  keywordstyle=\color{blue}
}
%\usepackage{booktabs}
\usepackage{caption}
\usepackage{subcaption}
\usepackage{float}
\usepackage{array}

\newcolumntype{M}[1]{>{\centering\arraybackslash}m{#1}}
\newcolumntype{N}{@{}m{0pt}@{}}


%%%%%%%%%%%%%%%%%%%%%%%%%%%%%%%%%%%%%%%%%%%%%%%%%%%%%%%%%%%%%%%%%%%%%%%%%%%%
%%%%%%%%%%%%%%%%%%%%%%%%%%%%%%%%%%%%%%%%%%%%%%%%%%%%%%%%%%%%%%%%%%%%%%%%%%%%
\newcommand{\itemEmail}{rcompanocca@unsa.edu.pe}
\newcommand{\itemStudent}{Roni Companocca Checco}
\newcommand{\itemCourse}{Programación}
\newcommand{\itemCourseCode}{20210558}
\newcommand{\itemSemester}{II}
\newcommand{\itemUniversity}{Universidad Nacional de San Agustín de Arequipa}
\newcommand{\itemFaculty}{Facultad de Ingeniería de Producción y Servicios}
\newcommand{\itemDepartment}{Departamento Académico de Ingeniería de Sistemas e Informática}
\newcommand{\itemSchool}{Escuela Profesional de Ingeniería de Sistemas}
\newcommand{\itemAcademic}{2023 - B}
\newcommand{\itemInput}{Del 29 Septiembre 2023}
\newcommand{\itemOutput}{Al 30 Septiembre 2023}
\newcommand{\itemPracticeNumber}{06}
\newcommand{\itemTheme}{HashMap}
%%%%%%%%%%%%%%%%%%%%%%%%%%%%%%%%%%%%%%%%%%%%%%%%%%%%%%%%%%%%%%%%%%%%%%%%%%%%
%%%%%%%%%%%%%%%%%%%%%%%%%%%%%%%%%%%%%%%%%%%%%%%%%%%%%%%%%%%%%%%%%%%%%%%%%%%%

\usepackage[english,spanish]{babel}
\usepackage[utf8]{inputenc}
\AtBeginDocument{\selectlanguage{spanish}}
\renewcommand{\figurename}{Figura}
\renewcommand{\refname}{Referencias}
\renewcommand{\tablename}{Tabla} %esto no funciona cuando se usa babel
\AtBeginDocument{%
	\renewcommand\tablename{Tabla}
}

\usepackage{fancyhdr}
\pagestyle{fancy}
\fancyhf{}
\setlength{\headheight}{30pt}
\renewcommand{\headrulewidth}{1pt}
\renewcommand{\footrulewidth}{1pt}
\fancyhead[L]{\raisebox{-0.2\height}{\includegraphics[width=3cm]{logo_episunsa.png}}}
\fancyhead[C]{\fontsize{7}{7}\selectfont	\itemUniversity \\ \itemFaculty \\ \itemDepartment \\ \itemSchool \\ \textbf{\itemCourse}}
\fancyhead[R]{\raisebox{-0.2\height}{\includegraphics[width=1.2cm]{abet.png}}}
\fancyfoot[L]{Estudiante Roni Companocca Checco}
\fancyfoot[C]{\itemCourse}
\fancyfoot[R]{Página \thepage}

% para el codigo fuente
\usepackage{listings}
\usepackage{color, colortbl}
\definecolor{dkgreen}{rgb}{0,0.6,0}
\definecolor{gray}{rgb}{0.5,0.5,0.5}
\definecolor{mauve}{rgb}{0.58,0,0.82}
\definecolor{codebackground}{rgb}{0.95, 0.95, 0.92}
\definecolor{tablebackground}{rgb}{0.8, 0, 0}

\lstset{frame=tb,
	language=bash,
	aboveskip=3mm,
	belowskip=3mm,
	showstringspaces=false,
	columns=flexible,
	basicstyle={\small\ttfamily},
	numbers=none,
	numberstyle=\tiny\color{gray},
	keywordstyle=\color{blue},
	commentstyle=\color{dkgreen},
	stringstyle=\color{mauve},
	breaklines=true,
	breakatwhitespace=true,
	tabsize=3,
	backgroundcolor= \color{codebackground},
}

\begin{document}
	
	\vspace*{10px}
	
	\begin{center}	
		\fontsize{17}{17} \textbf{ Informe de Laboratorio \itemPracticeNumber}
	\end{center}
	\centerline{\textbf{\Large Tema: \itemTheme}}
	%\vspace*{0.5cm}	

	\begin{flushright}
		\begin{tabular}{|M{2.5cm}|N|}
			\hline 
			\rowcolor{tablebackground}
			\color{white} \textbf{Nota}  \\
			\hline 
			     \\[30pt]
			\hline 			
		\end{tabular}
	\end{flushright}	

	\begin{table}[H]
		\begin{tabular}{|x{4.7cm}|x{4.8cm}|x{4.8cm}|}
			\hline 
			\rowcolor{tablebackground}
			\color{white} \textbf{Estudiante} & \color{white}\textbf{Escuela}  & \color{white}\textbf{Asignatura}   \\
			\hline 
			{\itemStudent \par \itemEmail} & \itemSchool & {\itemCourse \par Semestre: \itemSemester \par Código: \itemCourseCode}     \\
			\hline 			
		\end{tabular}
	\end{table}		
	
	\begin{table}[H]
		\begin{tabular}{|x{4.7cm}|x{4.8cm}|x{4.8cm}|}
			\hline 
			\rowcolor{tablebackground}
			\color{white}\textbf{Laboratorio} & \color{white}\textbf{Tema}  & \color{white}\textbf{Duración}   \\
			\hline 
			\itemPracticeNumber & \itemTheme & 02 horas   \\
			\hline 
		\end{tabular}
	\end{table}
	
	\begin{table}[H]
		\begin{tabular}{|x{4.7cm}|x{4.8cm}|x{4.8cm}|}
			\hline 
			\rowcolor{tablebackground}
			\color{white}\textbf{Semestre académico} & \color{white}\textbf{Fecha de inicio}  & \color{white}\textbf{Fecha de entrega}   \\
			\hline 
			\itemAcademic & \itemInput &  \itemOutput  \\
			\hline 
		\end{tabular}
	\end{table}

    \section{TAREA}
	\begin{itemize}	
    \subsection{Objetivos:}
		\item Crear e inicializar HashMaps
		\item Realizar búsquedas secuencial y binaria en un HashMap
        \item Implementar métodos de ordenamiento en HashMap
        \item Solucionar problemas
    \subsection{Competencias a alcanzar:}
		\item Diseña, responsablemente, sistemas, componentes o procesos para satisfacer necesidades dentro de restricciones realistas: económicas, medio ambientales, sociales, políticas, éticas, de salud, de seguridad, manufacturación y sostenibilidad.
        \item Aplica de forma flexible, técnicas, métodos, principios, normas, estándares y herramientas de ingeniería necesarias para la construcción de software e implementación de sistemas de información.

    \section{EQUIPOS, MATERIALES Y TEMAS UTILIZADOS}
	\begin{itemize}
		\item Sistema Operativo Windows
		\item OpenJDK 64-Bits 17.0.7.
		\item Git 2.39.2.	
  	\item Cuenta en GitHub con el correo institucional.
	\end{itemize}

    \section{URL DE REPOSITORIO GITHUB}
	\begin{itemize}
		\item URL para el Repositorio GitHub.
		\item \url{https://github.com/RONI-COMPANOCCA-CHECCO}
		\item URL para el laboratorio 06 en el Repositorio GitHub.	
        \item \url{https://github.com/RONI-COMPANOCCA-CHECCO/FP2-LAB6}
	\end{itemize}
    
    \section{ACTIVIDADES}
	\begin{itemize}
        \subsection{EJERCICIO RESUELTO}
		\item - Creamos la Clase ColeccionesMap con los siguientes métodos que realizan las operaciones que se pueden realizar con el HashMap:

        \begin{lstlisting}[language=java]
// RONI COMPANOCCA CHECCO
// CUI: 20210558
// LABORATORIO 06
// FUNDAMENTOS DE PROGRAMACION 
// EJERCICIO PROPUESTO DESARROLLADO
import java.util.Map;
import java.util.HashMap;
import java.util.Iterator;
import java.util.Scanner;

public class ColeccionesMap{
    static Scanner consola = new Scanner(System.in);
    // interfaz implement
    Map diccionario1 = new HashMap();
    public void llenarHashMap(){
        // Insertar valores "key"-"value" al HashMap
        diccionario1.put("29458865","Maria");
        diccionario1.put("28784599","Jose");
        diccionario1.put("26558899","Luis");
        diccionario1.put("45785214","Juan");
        diccionario1.put("48889566","Mariela");
        diccionario1.put("48889544","Carlos");
        diccionario1.put("12389566","Rosa");
        diccionario1.put("48889540","Carlos");
        diccionario1.put("26558890","Luis");

        // definir iterator ara extraer o imprimir valores
        // recorremos y mostramos el diccionario
        for(Iterator iterador1 = diccionario1.keySet().iterator(); iterador1.hasNext();){
            String key = (String)iterador1.next();
            String value = (String)diccionario1.get(key);
            System.out.println("DNI: "+key+" Alumno: "+value);
        }
        // mstramos el tamaño del diccionario
        System.out.println("Tamaño: "+diccionario1.size());
    }

    public void buscarEnHashMap(){
        // buscamos un valor en el diccionario
        System.out.println("Ingrese un Nombre a Buscar: ");
        String nombre = consola.next();
        if(diccionario1.containsValue(nombre)){
            System.out.println((nombre+" se encuentra en el Diccionario"));
        }else{
            System.out.println(nombre+ " No se encuentra!!");
        }
    }

    public void eliminarEnHashMap(){
        // eliminamos un valor en el diccionario
        System.out.println("Ingrese un Codigo a Eliminar: ");
        String llave = consola.next();
        diccionario1.remove( llave);

        for(Object key : diccionario1.keySet()){
            String llave2 = (String)key;
            String value2 = (String)diccionario1.get(llave2);
            System.out.println("DNI: "+llave2+" Alumno: "+value2);
        }
    }

    public void reemplazarEnHashMap(){
        // reemplazamos un valor en el diccionario
        System.out.println("Ingrese un codigo a reemplazar: ");
        String llave2 = consola.next();
        System.out.println("Ingrese el Nuevo Valor: ");
        String valorNuevo = consola.next();
        diccionario1.replace(llave2, valorNuevo);
        
        for(Object key3 : diccionario1.keySet()){
            String llave3 = (String)key3;
            String value3 = (String)diccionario1.get(llave3);
            System.out.println(("DNI: "+llave3+" Alumno: "+value3));
        } 
    }
}
        \end{lstlisting}
		\item Creamos la Clase Main, donde llamamos a nuestra clase y las operaciones implementadas:
        \begin{lstlisting}[language=java]
public class Map01 {
    public static void main(String[] args){
        ColeccionesMap coleccion01 = new ColeccionesMap();
        System.out.println("----------------------");
        System.out.println("COLECCION HASHMAP");
        System.out.println("----------------------");
        coleccion01.llenarHashMap();
        coleccion01.buscarEnHashMap();
        coleccion01.reemplazarEnHashMap();
        coleccion01.eliminarEnHashMap();
    }
}
        \end{lstlisting}
        \item Ejecucion: Ejecutamos el ejercicio desarrollado por la cual nos como respuesta lo siguiente.
        \begin{lstlisting}[language=java]
----------------------
COLECCION HASHMAP
----------------------
DNI: 26558899 Alumno: Luis
DNI: 48889540 Alumno: Carlos
DNI: 28784599 Alumno: Jose
DNI: 29458865 Alumno: Maria
DNI: 48889566 Alumno: Mariela
DNI: 48889544 Alumno: Carlos
DNI: 12389566 Alumno: Rosa
DNI: 26558890 Alumno: Luis
DNI: 45785214 Alumno: Juan
Tamaño: 9
Ingrese un Nombre a Buscar:
Carlos
Carlos se encuentra en el Diccionario
Ingrese un codigo a reemplazar:
48889544
Ingrese el Nuevo Valor:
Roni
DNI: 26558899 Alumno: Luis
DNI: 48889540 Alumno: Carlos
DNI: 28784599 Alumno: Jose
DNI: 29458865 Alumno: Maria
DNI: 48889566 Alumno: Mariela
DNI: 48889544 Alumno: Roni
DNI: 12389566 Alumno: Rosa
DNI: 26558890 Alumno: Luis
DNI: 45785214 Alumno: Juan
Ingrese un Codigo a Eliminar:
48889544
DNI: 26558899 Alumno: Luis
DNI: 48889540 Alumno: Carlos
DNI: 28784599 Alumno: Jose
DNI: 29458865 Alumno: Maria
DNI: 48889566 Alumno: Mariela
DNI: 12389566 Alumno: Rosa
DNI: 26558890 Alumno: Luis
DNI: 45785214 Alumno: Juan
        \end{lstlisting}

        \subsection{EJERCICIO PROPUESTO}
        \item Cree un Proyecto llamado Laboratorio6
        \item Usted deberá crear las dos clases Soldado.java y VideoJuego5.java. Puede reutilizar lo desarrollado en Laboratorios anteriores.
        \item Del Soldado nos importa el nombre, puntos de vida, fila y columna (posición en el tablero).
        \item El juego se desarrollará en el mismo tablero de los laboratorios anteriores. Para crear el tablero utilice la estructura de datos más adecuada.
        \item Tendrá 2 Ejércitos (usar HashMaps). Inicializar el tablero con n soldados aleatorios entre 1 y 10 para cada Ejército. Cada soldado tendrá un nombre autogenerado: Soldado0X1, Soldado1X1, etc., un valor de puntos de vida autogenerado aleatoriamente [1..5], la fila y columna también autogenerados aleatoriamente (no puede haber 2 soldados en el mismo cuadrado). Se debe mostrar el tablero con todos los soldados creados (distinguir los de un ejército de los del otro ejército). Además de los datos del Soldado con mayor vida de cada ejército, el promedio de puntos de vida de todos los soldados creados por ejército, los datos de todos los soldados por ejército en el orden que fueron creados y un ranking de poder de todos los soldados creados por ejército (del que tiene más nivel de vida al que tiene menos) usando 2 diferentes algoritmos de ordenamiento (indicar conclusiones respecto a este ordenamiento de HashMaps). Finalmente, que muestre qué ejército ganará la batalla (indicar la métrica usada para decidir al ganador de la batalla). Hacerlo como programa iterativo.
        
        \item la clase VideoJuego5.java        
        \begin{lstlisting}[language=java]
// RONI COMPANOCCA CHECCO
// CUI: 20210558
// LABORATORIO 06 - HASHMAP
// FUNDAMENTOS DE PROGRAMACION 

        \end{lstlisting}

        \item la clase Soldado.java
        \begin{lstlisting}[language=java]
// RONI COMPANOCCA CHECCO
// CUI: 20210558
// LABORATORIO 06
// FUNDAMENTOS DE PROGRAMACION - LABORATORIO
         \end{lstlisting}

         \item Ejecucion
         \begin{lstlisting}[language=java]

         \end{lstlisting}
 
    \section{CUESTIONARIO}

	\begin{itemize}
	\subsection {¿Cómo se declara e inicializa un HashMap?}
        \item Para declarar e inicializar un HashMap en Java
        \begin{lstlisting}[language=java]
        import java.util.HashMap;
        HashMap<KeyType, ValueType> nombreDelMapa = new HashMap<>();
        \end{lstlisting}
		
    \subsection {¿Qué ventajas tienen los HashMap con respecto a los arreglos y ArrayList?}
        \item Los HashMap en Java y otras estructuras de datos basadas en tablas de hash ofrecen varias ventajas en comparación con los arreglos (arrays) y ArrayLists como: Búsqueda eficiente, flexibilidad en las claves, capacidad dinámica, eliminación eficiente, no hay restricciones en la ubicación, claves únicas, etc.

	\subsection {¿Mencione 4 métodos importantes del HashMap? ¿Qué finalidad tienen?}
        \item put(Key key, Value value): Agregar un par clave-valor al HashMap. Si la clave ya existe en el mapa, el valor asociado se actualiza con el nuevo valor proporcionado. Si la clave no existe, se crea un nuevo par clave-valor en el mapa.
        \item get(Object key): Recuperar el valor asociado con una clave específica en el HashMap.
        \item remove(Object key): Eliminar el par clave-valor asociado con la clave proporcionada del HashMap.
        \item containsKey(Object key): Verificar si el HashMap contiene una clave específica. Devuelve true si la clave existe en el mapa y false en caso contrario.
	\end{itemize}
	
	\section{REFERENCIAS}
	\begin{itemize}
		\item M. Aedo, “Fundamentos de Programación 2 - Tópicos de Programación Orientada a Objetos”, Primera Edición, 2021, Editorial UNSA.
		\item \url{https://github.com/rescobedoq/programacion.git}
		\item J. Dean, "Introduction to programming with Java: A Problem Solving Approach”, Third Edition, 2021, McGraw-Hill.
        \item C. T. Wu, "An Introduction to Object-Oriented Programming with Java", Fifth Edition, 2010, McGraw-Hill.
        \item P. Deitel, "Java How to Program", Eleventh Edition, 2017, Prentice Hall.
	\end{itemize}
	
%\clearpage
%\bibliographystyle{apalike}
%\bibliographystyle{IEEEtranN}
%\bibliography{bibliography}
			
\end{document}